% Vim + SageTeX Interactive Workshop
% Copyright (C) 2024 microncomputer
%
% This workshop material is free software: you can redistribute it and/or modify
% it under the terms of the GNU General Public License as published by
% the Free Software Foundation, either version 3 of the License, or
% (at your option) any later version.
%
% This material is distributed in the hope that it will be useful,
% but WITHOUT ANY WARRANTY; without even the implied warranty of
% MERCHANTABILITY or FITNESS FOR A PARTICULAR PURPOSE. See the
% GNU General Public License for more details.
%
% You should have received a copy of the GNU General Public License
% along with this material. If not, see <https://www.gnu.org/licenses/>.

\documentclass[11pt]{article}
\usepackage{amsmath, amssymb, amsthm}
\usepackage{geometry}
\usepackage{xcolor}
\usepackage{listings}
\usepackage{hyperref}
\usepackage{sagetex}

\geometry{margin=1in}

% Define colors for vim commands
\definecolor{vimcolor}{RGB}{0,100,0}
\definecolor{exercisecolor}{RGB}{0,0,139}

% Custom commands for the workshop
\newcommand{\vimcmd}[1]{{\color{vimcolor}\texttt{#1}}}
\newcommand{\exercise}[1]{{\color{exercisecolor}\textbf{Exercise #1:}}}

\title{Interactive Workshop: Vim + SageMath + SageTeX}
\author{Mathematics Department Workshop}
\date{\today}

\begin{document}

\maketitle

\section*{Workshop Overview}
Today we'll learn to efficiently edit LaTeX documents using Vim while integrating computational mathematics through SageTeX. Keep this document open in Vim and follow along!

%==============================================================================
% VIM EXERCISE 1: Basic Navigation
% Try these commands while in normal mode:
% h, j, k, l - move cursor left, down, up, right
% w, b - move forward/backward by word
% 0, $ - move to beginning/end of line
% gg, G - move to beginning/end of file
%==============================================================================

\section{Part 1: Vim Basics for LaTeX Editing}

\subsection{Essential Vim Commands}

\exercise{1} Navigate to this line using \vimcmd{/Exercise 1} (search), then:
\begin{itemize}
    \item Delete this line using \vimcmd{dd}
    \item FIXME: This line has an error - fix it using \vimcmd{cw} (change word)
    \item Duplicate this line using \vimcmd{yy} then \vimcmd{p}
\end{itemize}

%==============================================================================
% VIM EXERCISE 2: Text Objects in LaTeX
% ci{ - change inside braces
% da$ - delete around dollar signs (math mode)
% vi[ - visually select inside brackets
%==============================================================================

\exercise{2} Practice with LaTeX-specific text objects:

Fix this equation: $x^2 + 2x + 1 = (x + 1)^{CHANGE_ME}$

Hint: Use \vimcmd{f\{} to find the brace, then \vimcmd{ci\{} to change inside braces.

\subsection{Efficient LaTeX Editing Patterns}

\exercise{3} Transform the following list using Vim macros:
% Record a macro with qa...q, then replay with @a

% START_LIST (transform these into \item entries)
Linear Algebra
Calculus III  
Abstract Algebra
Real Analysis
% END_LIST

% Your macro should:
% 1. qa (start recording)
% 2. 0i\item <Esc> (insert \item at beginning)
% 3. j (go to next line)
% 4. q (stop recording)
% 5. 3@a (replay 3 times)

%==============================================================================
% VIM EXERCISE 4: Search and Replace
% :%s/old/new/g - replace all occurrences
% :5,10s/old/new/g - replace in lines 5-10
%==============================================================================

\section{Part 2: Introduction to SageTeX}

\subsection{Basic Computations}

\exercise{4} Uncomment and modify the following SageTeX computations:

% \begin{sagesilent}
% # Define variables for reuse
% a = 5
% b = 12
% c = sqrt(a**2 + b**2)
% \end{sagesilent}

The Pythagorean triple: $a = \sage{a}$, $b = \sage{b}$, $c = \sage{c}$

\exercise{5} Create a multiplication table using Sage:

\begin{sagesilent}
# Create a 5x5 multiplication table
n = 5
mult_table = [[i*j for j in range(1, n+1)] for i in range(1, n+1)]
\end{sagesilent}

% Use Vim to duplicate and modify this line for different sizes
Multiplication Table ($\sage{n} \times \sage{n}$):

\[
\sage{matrix(mult_table)}
\]

%==============================================================================
% VIM EXERCISE 5: Visual Block Mode
% Ctrl+V - enter visual block mode
% I - insert at beginning of block
% A - append at end of block
%==============================================================================

\subsection{Calculus with Sage}

\exercise{6} Add derivative and integral using visual block editing:

\begin{sagesilent}
f(x) = x^3 - 3*x^2 + 2*x
df = diff(f, x)
intf = integrate(f, x)
critical_points = solve(df == 0, x)
\end{sagesilent}

% Add % signs at the beginning of these lines using Ctrl+V
Function: $f(x) = \sage{f(x)}$
Derivative: $f'(x) = \sage{df}$
Integral: $\int f(x)\,dx = \sage{intf} + C$
Critical points: $\sage{critical_points}$

\section{Part 3: Advanced Examples}

\subsection{Plotting with SageTeX}

\exercise{7} Modify plot parameters using Vim's substitute command:

\begin{sagesilent}
# Change these parameters using Vim
xmin, xmax = -3, 3
ymin, ymax = -2, 5
plot_color = 'blue'  # Try: red, green, purple

p = plot(f(x), (x, xmin, xmax), 
         color=plot_color, 
         thickness=2,
         title='Cubic Function')
\end{sagesilent}

\sageplot[width=0.5\textwidth]{p}

\subsection{Linear Algebra Computations}

\exercise{8} Create and manipulate matrices:

\begin{sagesilent}
# Define a matrix
A = matrix([[1, 2, 3], 
            [4, 5, 6], 
            [7, 8, 10]])
            
det_A = A.det()
inv_A = A.inverse()
eigen = A.eigenvalues()
\end{sagesilent}

Matrix $A = \sage{A}$

% Practice: Use . command to repeat last edit
Determinant: $\det(A) = \sage{det_A}$
% ADD_LINE_HERE: Inverse matrix
% ADD_LINE_HERE: Eigenvalues

\subsection{Number Theory Example}

\exercise{9} Explore prime numbers:

\begin{sagesilent}
def goldbach_pairs(n):
    """Find all pairs of primes that sum to n"""
    if n % 2 != 0 or n < 4:
        return []
    pairs = []
    for p in prime_range(2, n//2 + 1):
        if is_prime(n - p):
            pairs.append((p, n-p))
    return pairs

test_number = 100
pairs = goldbach_pairs(test_number)
\end{sagesilent}

Goldbach pairs for \sage{test_number}: \sage{pairs}

%==============================================================================
% VIM EXERCISE 6: Registers and Marks
% "ayy - yank line into register a
% "ap - paste from register a
% ma - set mark a
% 'a - jump to mark a
%==============================================================================

\section{Part 4: Mini-Project}

\exercise{10} Complete this template for a mathematical exploration:

\begin{sagesilent}
# YOUR TOPIC: Fibonacci Sequence Analysis
# TODO: Complete this code using Vim

def fibonacci(n):
    """Generate first n Fibonacci numbers"""
    fib = [0, 1]
    # Use Vim to complete this function
    # Hint: Use >> to indent, << to dedent
    return fib[:n]

# Analyze the sequence
fib_sequence = fibonacci(10)
# golden_ratio = 
# ratio_convergence = 
\end{sagesilent}

% Create your analysis here using Vim and SageTeX

\section{Vim Cheat Sheet for LaTeX}

\begin{tabular}{ll}
\hline
\textbf{Command} & \textbf{Description} \\
\hline
\vimcmd{:w} & Save file \\
\vimcmd{:!pdflatex \%} & Compile current file \\
\vimcmd{:!sage \%} & Process SageTeX \\
\vimcmd{/\\section} & Search for sections \\
\vimcmd{ci\{} & Change inside braces \\
\vimcmd{dat} & Delete around tag \\
\vimcmd{gq} & Format paragraph \\
\vimcmd{\%} & Jump between matching delimiters \\
\vimcmd{:set spell} & Enable spell checking \\
\vimcmd{z=} & Suggest spelling corrections \\
\hline
\end{tabular}

\section{Compilation Workflow}

To compile this document with SageTeX:
\begin{enumerate}
    \item \texttt{pdflatex workshop.tex} (initial compilation)
    \item \texttt{sage workshop.sagetex.sage} (process Sage code)
    \item \texttt{pdflatex workshop.tex} (final compilation)
\end{enumerate}

In Vim, you can create a macro for this:
\begin{verbatim}
:!pdflatex % && sage %.sagetex.sage && pdflatex %
\end{verbatim}

\section*{Challenge Problems}

\exercise{Challenge 1} Use Vim macros to create a table of derivatives:

% Start with these functions and create a formatted table
% f(x) = x^n for n = 1, 2, 3, 4, 5

\exercise{Challenge 2} Create a SageTeX visualization of the Newton-Raphson method:

\begin{sagesilent}
# Implement Newton's method here
# def newton_method(f, df, x0, iterations):
#     pass
\end{sagesilent}

\exercise{Challenge 3} Use Vim's visual block mode to create a matrix in LaTeX:

% Transform this into a proper LaTeX matrix:
% 1 0 0 0
% 0 1 0 0  
% 0 0 1 0
% 0 0 0 1

\section*{Additional Resources}

\begin{itemize}
    \item Vim Adventures: \url{https://vim-adventures.com/}
    \item SageMath Documentation: \url{https://doc.sagemath.org/}
    \item SageTeX Tutorial: \url{https://doc.sagemath.org/html/en/tutorial/sagetex.html}
    \item LaTeX Wikibook: \url{https://en.wikibooks.org/wiki/LaTeX}
\end{itemize}

\section*{Workshop Feedback}

% Use Vim to quickly comment/uncomment these lines
Rate your comfort level (1-5):
\begin{itemize}
    \item Vim navigation: \_\_\_
    \item Vim text objects: \_\_\_
    \item SageTeX basics: \_\_\_
    \item Combining both tools: \_\_\_
\end{itemize}

\end{document}
