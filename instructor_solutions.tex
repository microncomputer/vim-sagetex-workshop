\documentclass[11pt]{article}
\usepackage{amsmath, amssymb}
\usepackage{geometry}
\usepackage{xcolor}
\usepackage{sagetex}

\geometry{margin=1in}
\title{Workshop Solutions - Instructor Guide}
\date{\today}

\begin{document}
\maketitle

\section*{Exercise Solutions}

\textbf{Exercise 3 - Vim Macro Solution:}

The macro transforms plain text to LaTeX list items:
\begin{verbatim}
qa               " Start recording macro 'a'
0                " Go to beginning of line  
i\item <Space>   " Insert \item and space
<Esc>            " Exit insert mode
j                " Move to next line
q                " Stop recording
3@a              " Replay macro 3 times
\end{verbatim}

Result:
\begin{itemize}
\item Linear Algebra
\item Calculus III  
\item Abstract Algebra
\item Real Analysis
\end{itemize}

\textbf{Exercise 4 - Basic SageTeX:}

\begin{sagesilent}
# Pythagorean triple example
a = 5
b = 12
c = sqrt(a**2 + b**2)
\end{sagesilent}

The Pythagorean triple: $a = \sage{a}$, $b = \sage{b}$, $c = \sage{c}$

\textbf{Exercise 7 - Visual Block Solution:}

To comment multiple lines:
\begin{verbatim}
1. Ctrl+V to enter visual block mode
2. Select the beginning of lines with j/k
3. I to insert at beginning
4. Type "% "
5. Esc to apply to all selected lines
\end{verbatim}

\textbf{Exercise 10 - Fibonacci Complete Solution:}

\begin{sagesilent}
def fibonacci(n):
    """Generate first n Fibonacci numbers"""
    if n <= 0:
        return []
    elif n == 1:
        return [0]
    elif n == 2:
        return [0, 1]
    else:
        fib = [0, 1]
        for i in range(2, n):
            fib.append(fib[i-1] + fib[i-2])
        return fib

# Analyze the sequence
fib_sequence = fibonacci(15)
golden_ratio = (1 + sqrt(5)) / 2
ratio_convergence = [float(fib_sequence[i+1]/fib_sequence[i]) 
                     for i in range(2, 14)]
\end{sagesilent}

Fibonacci sequence (first 15): \sage{fib_sequence}

Golden ratio: $\phi = \sage{golden_ratio.n(digits=6)}$

Convergence to golden ratio:
\sage{[round(r, 6) for r in ratio_convergence[-5:]]}

\section*{Challenge Solutions}

\textbf{Challenge 1 - Derivative Table Macro:}

Record this macro:
\begin{verbatim}
qa
yy5p                  " Copy line and paste 5 times
:%s/n/1/g            " Replace n with 1,2,3,4,5 manually
q
\end{verbatim}

\begin{sagesilent}
powers = range(1, 6)
\end{sagesilent}

\begin{tabular}{|c|c|}
\hline
$f(x)$ & $f'(x)$ \\
\hline
$x$ & $1$ \\
$x^2$ & $2x$ \\
$x^3$ & $3x^2$ \\
$x^4$ & $4x^3$ \\
$x^5$ & $5x^4$ \\
\hline
\end{tabular}

\textbf{Challenge 2 - Newton's Method:}

\begin{sagesilent}
def newton_method(f, df, x0, iterations):
    """Implement Newton-Raphson method"""
    points = [x0]
    x = x0
    for i in range(iterations):
        x = x - f(x)/df(x)
        points.append(float(x))
    return points

# Example: finding sqrt(2)
f(x) = x^2 - 2
df(x) = 2*x
x0 = 1.5
iterations = 5
results = newton_method(f, df, x0, iterations)

# Create visualization
p = plot(f, (x, 0, 2.5), color='blue', thickness=2)
p += line([(0,0), (2.5,0)], color='black', linestyle='--')
p += point((sqrt(2), 0), size=40, color='red')

# Add tangent lines
for i in range(2):
    xi = results[i]
    yi = f(xi)
    slope = df(xi)
    tangent = lambda t: yi + slope*(t - xi)
    p += plot(tangent, (x, max(0, xi-0.5), min(2.5, xi+0.5)), 
              color='green', thickness=1, alpha=0.5)
    p += point((xi, yi), size=30, color='orange')
\end{sagesilent}

Newton's Method for $f(x) = x^2 - 2$:

Iterations from $x_0 = \sage{x0}$:
\begin{itemize}
\item $x_0 = \sage{round(results[0], 6)}$
\item $x_1 = \sage{round(results[1], 6)}$
\item $x_2 = \sage{round(results[2], 6)}$
\item $x_3 = \sage{round(results[3], 6)}$
\item $x_4 = \sage{round(results[4], 6)}$
\end{itemize}

Actual value: $\sqrt{2} = \sage{sqrt(2).n(digits=6)}$

\sageplot[width=0.6\textwidth]{p}

\textbf{Challenge 3 - Visual Block Matrix:}

Steps to create LaTeX matrix:
\begin{verbatim}
1. Visual block select all numbers (Ctrl+V)
2. Add & between columns
3. Add \\ at end of lines
4. Surround with \begin{pmatrix} ... \end{pmatrix}
\end{verbatim}

Result:
\[
\begin{pmatrix}
1 & 0 & 0 & 0 \\
0 & 1 & 0 & 0 \\  
0 & 0 & 1 & 0 \\
0 & 0 & 0 & 1
\end{pmatrix}
\]

\section*{Teaching Notes}

\subsection*{Common Student Difficulties}

\begin{enumerate}
\item \textbf{Mode confusion}: Students often forget which mode they're in
   \begin{itemize}
   \item Solution: Encourage frequent Esc pressing
   \item Show them :set showmode
   \end{itemize}

\item \textbf{Sage compilation errors}: Missing semicolons or Python syntax
   \begin{itemize}
   \item Solution: Check .sagetex.sage file directly
   \item Run sage interactively to test code
   \end{itemize}

\item \textbf{Visual block selection}: Getting exact selection
   \begin{itemize}
   \item Solution: Use o to switch corners
   \item Practice with simple examples first
   \end{itemize}
\end{enumerate}

\subsection*{Time Management Tips}

\begin{itemize}
\item Have students pair up for exercises
\item Pre-compile the document to save time
\item Keep a "rescue" version ready
\item Use screen sharing for demonstrations
\item Have cheat sheets printed and ready
\end{itemize}

\end{document}
